\documentclass{article}
\usepackage{graphicx} % Required for inserting images
\usepackage{indentfirst}
\usepackage{mathtools}
\usepackage{amsthm}
\usepackage{amssymb}
\usepackage{amsmath}

\theoremstyle{plain}
\newtheorem{veta}{Věta}[section]

\theoremstyle{definition}
\newtheorem{definice}[veta]{Definice}

\newtheorem{lemma}[veta]{Lemma}
\newtheorem*{znaceni}{Značení}
\newtheorem{algoritmus}{Algoritmus}

\title{Bakalářská práce WIP}
\author{Benjamin de Silva}
\date{October 2023}

\begin{document}

\maketitle
\newpage
\section*{Úvod}
Motivace
\newpage

\section{Definice (hlavní zdroj - linalg skripta)}
V této kapitole si zavedeme základní definice potřebné pro práci s maticemi, operace na nich a různé typy matic, které budeme používat.

\begin{definice}[Matice]

\emph{Maticí} typu m $\times$ n nad tělesem \textbf{T} rozumíme obdelníkové schéma prvků z tělesa \textbf{T} s m řádky a n sloupci. Matice typu m $\times$ m se nazývá \emph{čtvercová matice řádu m}. Matice typu m $\times$ 1 se nazýva \emph{sloupcový aritmetický vektor} a matice typu 1 $\times$ m se nazývá \emph{řádkový aritmetický vektor}. Zápisem $A=(a_{ij})_{m\times n}$ rozumíme matici typu $m\times n$ s prvkem $a_{ij}\in T$ na pozici (i,j).
\end{definice}
    V rámci této práce budeme uvažovat matice nad tělesem $T=\mathbb{C}$

\begin{definice}[Typy matic]
Nechť A je čtvercová matice řádu n. Pak řekneme, že
\begin{itemize}
    \item A je \emph{nulová}, pokud $a_{ij}=0, \forall i,j \in\{1,...,n\}$,
    \item A je \emph{jednotková}, pokud $a_{ij}=0$ pro $i=j$ a $a_{ij}=0$ jinak,
    \item A je \emph{diagonální}, pokud $a_{ij}=0$ pro $i\neq j$,
    \item A je \emph{horní trojuhelníková}, pokud $a_{ij}=0$ pro $i>j$,
    \item A je \emph{dolní trojuhelníková}, pokud $a_{ij}=0$ pro $i<j$,
    \item A je \emph{permutační}, pokud má v každém řádku a sloupci právě jeden prvek 1 a zbytek jsou nuly,
    \item A je \emph{symetrická}, pokud $a_{ij}=a_{ji}\;, \forall i,j \in\{1,...,n\}$,
    \item A je \emph{řídká}, pokud většina prvků matice je nulových.
\end{itemize}
\end{definice}

\begin{znaceni}
    Budeme značit jednotkovou matici $\mathbf{I_{m \times n}}$ a nulovou matici $\mathbf{0_{m \times n}}$, jsou-li obě matice typu $m\times n $. Pokud jsou matice čtvercové řádu n, tak značíme $\mathbf{I_n, 0_n}$ a v případě, že je to z kontextu zřejmé, tak index může být vynechán.
\end{znaceni}

\begin{definice}[Sčítání matic]
Nechť A,B jsou matice stejného typu m $\times$ n, pak definujeme jejich součet jako matici $$A+B=(a_{ij}+b_{ij})_{m \times n}$$
\end{definice}
Nechť A,B,C jsou matice stejného typu, pak operace sčítání matic má následující vlastnosti
\begin{itemize}
    \item $(A+B)+C=A+(B+C)$,
    \item  $A+0=A$,
    \item $A+(-A)=0$,
    \item $A+B=B+A$.
    
\end{itemize}
\newpage

\begin{definice}[Násobení matice skalárem]
Nechť A je matice typu $m \times n$ a $t \in \mathbb{R}$. Pak definujeme t-násobek matice A jako matici $$t\cdot A = tA = (t \cdot a_{ij})_{m \times n}$$
    
\end{definice}
Nechť A,B jsou matice stejného typu a $s,t \in \mathbb{R}$. Pak platí následující vlastnosti
\begin{itemize}
    \item $s(tA)=(st)A$,
    \item $1A=A$,
    \item $-A=(-1)A$,
    \item $(s+t)A=sA+tA$,
    \item $s(A+B)=sA+sB$.
\end{itemize}

\begin{definice}[Transponovaná matice]
Nechť A je matice typu $m \times n$, pak  transponovanou maticí k matici A rozumíme matici $$A^T=(b_{ij})_{n\times m},$$ kde $b_{ij}=a_{ji}$ pro $i \in \{1,\hdots,n\}, j\in \{1,\hdots,m\}$.
\end{definice}
Z definice symetrické matice a transponované matice plyne, že pro A symetrickou platí $A^T=A$.

Nechť A,B jsou matice stejného typu $m\times n$ a $s \in \mathbb{C}$, pak platí
\begin{itemize}
    \item $(A^T)^T=A$
    \item $(A^T+B^T)=A^T+B^T$
    \item $(sA)^T=sA^T$
\end{itemize}

\begin{znaceni}
Někdy je vhodné nahlížet na matici jako na posloupnost sloupcových vektorů, tedy matici $A=(a_{ij})_{m \times n}$ můžeme zapsat jako $A=( \mathbf{a_1}|\mathbf{a_2}|...|\mathbf{a_n})$, kde $\mathbf{a_i}=(a_{1i},a_{2i},\hdots,a_{mi})^T,i \in \{1,\hdots,n\}$ je i-tý m-složkový sloupcový aritmetický vektor matice A,
\end{znaceni}

\begin{definice}[Součin matice a vektoru]
Nechť $A=( \mathbf{a_1}|\mathbf{a_2}|...|\mathbf{a_n})$ je matice typu $m \times n$ a $b=(b_1,b_2,\hdots,b_n)^T$ je aritmetický sloupcový vektor. Pak definujeme součin matice A s vektorem b jako $$A\cdot b= b_1 \mathbf{a_1} + b_2 \cdot \mathbf{a_2}+\hdots+b_n\mathbf{a_n},$$ což je m-složkový sloupcový vektor.
\end{definice}


\begin{definice}[Součin dvou matic]
Nechť A je matice typu $m \times n$ a $B=(\mathbf{b_1},\hdots,\mathbf{b_k})$ je matice typu $n \times k$. Pak součin matic A a B je $$AB=(A\mathbf{b_1},A\mathbf{b_2},\hdots,A\mathbf{b_k}),$$ což je matice typu $m \times k$.
\end{definice}
Ze zavedených definic už můžeme začít řešit úlohu soustav lineárních rovnic.
Mějme zadanou matici A typu $m \times n$ a $b=(\mathbf{b_1},\hdots,\mathbf{b_m})^T$ m-složkový aritmetický vektor. Pak hledáme $x=(\mathbf{x_1},\hdots,\mathbf{x_n})^T$ sloupcový vektor takový, že platí rovnost $\mathbf{Ax=b}$. Každý takový vektor $\mathbf{x}$ nazveme řešením soustavy linearních rovnic matice A s pravou stranou $\mathbf{b}$.

\begin{definice}[Inverzní matice]
Nechť A je matice typu $m \times n $. Pak
\begin{itemize}
    \item matici X nazveme inverzní k A zprava, pokud platí $AX=I_m$
    \item matici X nazveme inverzní k A zleva, pokud platí $AX=I_n$
    \item je-li A čtvercová řádu n a matice X taková, že $AX=XA=I_n$, pak X nazveme inverzní k matici A, matici X značíme $A^{-1}$ a matici A nazveme invertibilním
\end{itemize}
\end{definice}

\begin{definice}[Regulární matice]
    Matici A řádu n nazveme regulární, pokud je zobrazení $f_A:\mathbb{C}^n \rightarrow \mathbb{C}^n,x\mapsto Ax$ určené maticí A bijektivní. Matice A, která není regulární, se nazývá singulární.
\end{definice}
    Pro čtvercovou matici A platí následující ekvivalence
\begin{itemize}
    \item Matice A je regulární,
    \item Zobrazení $f_A$ je na
    \item Zobrazení $f_A$ je prosté
    \item Soustava $A\mathbf{x}=0$ má právě jedno řešení $\mathbf{x}=0$
    \item Matice A je invertovatelná
    \item Existuje matice X tak, že $AX=I_n$
    \item Existuje matice Y tak, že $YA=I_n$
    \item $rank(A)=n$
\end{itemize}
\begin{definice}[Obor hodnot, jádro, hodnost]
Pro matici A typu $m \times n$ definujeme $$\mathbf{R}(A)=\{A\mathbf{x}:x\in \mathbb{R}^n\}\;, \;\mathbf{N}(A)=\{\mathbf{x}\in\mathbf{C}^n\}\;,\; rank(A)\equiv dim(\mathbf{C}(A),$$ kde $R(A)$ nazveme \emph{oborem hodnot} matice A, $\mathbf{N}(A)$ nazveme \emph{jádrem} matice A a rank(A) \emph{hodností} matice A. 
    
\end{definice}

\newpage

\section{Metody pro hledání nulového prostoru (zdroj-kniha se stromem}
V této kapitole popíšeme různé metody pro hledání nulového prostoru matic. Mějme tedy nějakou matici A typu $m \times n$ řádu m, pak hledáme nějakou matici B typu $n \times (n-m)$ řádu $n-m$ tak, že platí $AB=0$, tedy sloupce matice B tvoří bázi nulového prostoru matice A.

Pro každou metodu bude výhodné nejprve rozložit matici A a poté matici B spočítáme za použití získaného rozkladu. Abychom zachovali numerickou stabilitu a řídkost, tak je někdy nutné prohodit některé řádky a sloupce za pomocí permutačních matic, tedy ve skutečnosti hledáme rozklad matice $P_1 E P_2$.

\subsection{Singulární rozklad}
Singulární rozklad (zkráceně SVD - singular value decomposition) nám umožňuje určit hodnost či normu matice, ortogonální bázi oboru hodnot a nulového prostoru matice, nicméně má vysoké výpočetní nároky. Před zavedením singulárního rozkladu si nejprve zavedeme spektrální rozklad.
\subsubsection{Spektrální rozklad}
\begin{veta}
Nechť $A \in \mathbb{C}^{m \times n}$, potom platí 
$$N(A) \bigoplus R(A^*) = \mathbb{C}^n,\;\;N(A) \perp R(A^*),$$
$$N(A^*) \bigoplus R(A) = \mathbb{C}^m,\;\;N(A^*) \perp R(A).$$
\end{veta}
\begin{veta}[Spektrální rozklad pro normální matice]
Matice A je normální právě tehdy, když existuje unitární matice U a diagonální matice D tak, že
$$U^{*}AU=D,\;\; tj. \;\;A=UDU^{*}$$
\end{veta}
Předpokládejme, že A je hermitovská pozitivně semidefinitní řádu n s hodností r s nezápornými vlastními čísly $\lambda_1 \geq \lambda_2 \geq \hdots \geq \lambda_r > 0$, $\lambda_{r+1}=\hdots=\lambda_n=0$. Podle věty 2.1 existuje rozklad matice A ve tvaru
$$A=Q\Lambda Q^*,\;\;Q^*Q=I,\;\;\Lambda=diag(\lambda_1,\hdots,\lambda_n).$$
Sloupce unitární matice $Q=(q_1,\hdots,q_n)$ jsou vlastní vektory matice A a platí
$$Aq_j=\lambda_jq_j,\;\;j=1,\hdots,n$$
Vlastní vektory $q_1,q_2,\hdots,q_n$ tvoří ortonormální bázi prostoru $\mathbb{C}^n$.

Ze spektrálního rozkladu matice A vidíme bázi nulového prostoru a oboru hodnot matice A, $$R(A)=span\{q_1,\hdots,q_r\},\;\;N(A)=span\{q_{r+1},\hdots,q_n\},$$ dále nám dává dyadický rozvoj matice A $$A=\sum_{j=1}^{n} \lambda_j q_j q_j^*=\sum_{j=1}^{m}A_j,\;\;A_j=\lambda_j q_j q_j^*,$$ kde matice $A_j$ jsou uspořádány podle spektrální (či Frobeniovy) normy matice a platí $\left \lVert A_j \right \rVert=\left \lVert A_j \right \rVert _F=\lambda_j$, tedy můžeme matici A aproximovat maticí nižší hodnosti $$A^{(k)}=\sum_{j=1}^k A_j.$$

Pro odvození singulárního rozkladu budou důležité spektrální rozklady matic $A^*A$ a $AA^*$. Matice $A^*A$ a $AA^*$ jsou čtvercové hermitovské pozitivně semidefinitní a můžeme uvažovat jejich spektrální rozklad.
\begin{lemma}
Pro komplexní matici $A \in \mathbb{C}^{m \times n}$ platí $$N(A^*A)=N(A),\;\;R(A^*A)=R(A^*),$$
$$N(AA^*)=N(A^*),\;\;R(AA^*)=R(A).$$
Důsledek \textbf{Věty 2.1} a \textbf{Lemma 2.3} je, že 
$$dim(R(AA^*))=dim(R(A))=r=dim(R(A^*))=dim(R(A^*A)),$$
pro A hodnosti r.
\end{lemma}

Buď $\lambda_j$ vlastní čísla a $v_j$ ortonormální vlastní vektory matice $A^*A$, $j=1,\hdots,n.$ Uvažujme dále bez újmy na obecnosti 
$$\lambda_1 \geq \hdots \geq \lambda_r > 0,\;\; \lambda_{r+1}=\lambda_n=0,$$
pak pro $V=(v_1,\hdots,v_n)$ platí 
$$V^*A^*AV=diag(\lambda_1,\hdots,\lambda_r,0,\hdots,0)$$
Dále z \textbf{Lemma 2.3} platí, že $v_1,\hdots,v_r$ resp. $v_{r+1},\hdots,v_n$ tvoří ortonormální bázi
$$R(A^*A)=R(A^*)=span\{v_1,\hdots,v_r\},$$resp. $$N(A)=N(A^*A)=span\{v_{r+1},\hdots,v_n\}.$$

Následující lemma nám dává vztah mezi ortonormálními vektory matic $A^*A$ a $AA^*$.
\begin{lemma}
Uvažujme spektrální rozklad matice $A^*A$ s vlastními čísly $\lambda_j$ a $v_j$, $j=1,\hdots,r.$ Potom jsou vektory
$$u_j \equiv Av_j/\sqrt{\lambda_j},\;\;j=1,\hdots,r$$
ortonormální vlastní vektory $AA^*$ a platí
$$AA^*v_j=\lambda_j u_j,\;\; \left \lVert u_j \right \rVert = 1,\;\; j=1,\hdots,r. $$
Z \textbf{Lemma 2.4} plyne, že vektory $u_j,\;j=1,\hdots,r$ jsou ortonormální vlastní vektory matice $AA^*$, které odpovídají právě m nenulovým vlastním číslům $\lambda_j$ a platí 
$$Av_j=\sqrt{\lambda_j}u_j,\;\;j=1,\hdots,r.$$

Analogie těchto vztahů platí i pro matici $AA^*$, tj. pokud $u_1,\hdots,u_r$ tvoří ortonormální bázi oboru hodnot matice a $u_{r+1},\hdots,u_m$ tvoří ortonormální bázi doplňku oboru hodnot, pak platí
$$span\{u_1,\hdots,u_r\}=R(AA^*)=R(A),$$
$$span\{u_{r+1},\hdots,u_m\}=N(AA^*)=N(A^*).$$
Z konstrukce plyne, že nenulová vlastní čísla matic $A^*A$ a $AA^*$ se rovnají včetně násobnosti.    
\end{lemma}

Nyní můžeme popsat singulární rozklad.
\begin{definice}[Singulární čísla]
Nechť $A\in\mathbb{C}^{m\times n}$. Odmocniny vlastních čísel matice $A^*A$ nazveme singulárnímy čísly matice A,
$$\sigma_j=\sqrt{\lambda_j}, \;\;j=1,\hdots,r,\;\;r=rank(A).$$
Jelikož vlastní čísla $\lambda_j$ jsou uspořádáná od největšího po nejmenší, tak platí
$$\sigma_1\geq\hdots\geq\sigma_r>0.$$ 
\end{definice}
Tedy podle předchozích poznatků platí $Av_j=\sigma_ju_j,\;j=1,\hdots,r$ a můžeme psát
$$AV=U\Sigma,$$
kde $$\Sigma=
    \begin{bmatrix}
    \Sigma_r & 0 \\
    0 & 0
    \end{bmatrix} \in \mathbb{R}^{m \times n},\;\; \Sigma_r=diag(\sigma_1,\hdots,\sigma_r)\in\mathbb{R}^{r\times r},\;\;,$$
    $$V=(v_1,\hdots,v_n). U=(u_1,\hdots,u_m).$$
Ekvivalentně
$$A=U\Sigma V^*,\;\;A^*=V\Sigma^TU^*.$$


Vidíme tedy, že pokud bychom byli schopni spočítat singulární rozklad matice A, tak nulový prostor podle \textbf{Lemma 2.3} je roven lineárnímu obalu vektorů $v_{r+1},\hdots,v_n.$
\newpage

\begin{veta}
    Mějme matici A $\in \mathbb{R}^{m \times n}$ řádu m a její singulární čísla $\sigma_1\geq\hdots\geq\sigma_m>0.$
Potom platí $\left \lVert A \right \rVert = \sigma_1,\;\;
\left \lVert A \right \rVert _F = (\sigma_1^2+\sigma_2^2+\hdots+\sigma_m^2)^{1/2}.$
Dále
$$\left \lVert A \right \rVert \leq \left \lVert A \right \rVert _F \leq \sqrt{m} \left \lVert A \right \rVert.$$
\end{veta}
Jelikož platí, že $$\left \lVert A \right \rVert = \sigma_1,\;a\; \left \lVert A^-1 \right \rVert = \sigma_m^-1,$$
tak dostáváme
$$\kappa(A)=\sigma_1 / \sigma_n.$$

\subsubsection{Výpočet nulového prostoru matice pomocí SVD}

Nyní pomocí SVD můžeme vyřešit úlohu hledání matice B splňující $AB=0$ následovně
$$P_1AP_2=U\Sigma V,$$
$$[V_1,V_2]=V^*,$$
$$B=P_2V_2.$$

SVD můžeme vypočítat za pomocí spektrálních rozkladů matic $AA^*$ a $A^*A$, ovšem podmíněnost obou matic je rovna $\kappa(A)^2$, tedy pokud by byla matice A špatně podmíněna, tak by podle \textbf{Věty 2.6.} mohl být výpočet singulárních čísel nepřesný.

Abychom se vyhnuli problémům se špatnou podmíněností matice, tak využíváme následující standardní metody pro výpočet SVD a následovně i nulový prostor matice A.

\begin{algoritmus}[Výpočet jádra matice pomocí SVD]
    Mějme matici A typu $m \times n$ hodnosti m.
    \begin{itemize}
        \item Transformujeme matici A na bidiagonální tvar užitím unitárních transformací nebo Golub-Kahanovy iterační bidiagonalizace. Transformovaná matice A je ve tvaru
        \[B=PAQ,\]
        kde P,Q jsou unitární a B je bidiagonální.
        \item Nalezení singulární rozkladu bidiagonální matice B pomocí iteračního procesu, který nuluje její naddiagonální prvky, například pomocí implicitního QR algoritmu.
        \item Máme rozklad $\Sigma=U^*AV$ a nulový prostor matice A tedy odpovídá sloupcovým vektorům $v_{m+1},\hdots,v_n$ matice V, tedy 
        \[B=[v_{m+1},\hdots,v_n].\]
    \end{itemize}
\end{algoritmus}
Algoritmus výše je proveden bez permutací, ovšem v obecném případě, kdy permutujeme matici A na matici $P_1AP_2$, tak můžeme opět aplikovat algoritmus výše a dostaneme rozklad
\[\Sigma=U^*P_1AP_2V,\]
a potom platí, že hledaná matice nulového prostoru je
\[B=P_2V_2.\]
Složitost výpočtu matice B závisí na iteračním procesu, kterým počítáme singulární rozklad bidiagonální matice. Kdybychom předpokládali, že A je čtvercová matice řádu m, iterační proces rychle konverguje a nezvyšuje počet operací, pak dolní odhad na počet operací je $\frac{16}{3}  m^3$.

\subsection{LU rozklad}
Jednou z nejdůležitejších metod pro řešení soustav lineárních rovnic je LU rozklad, který nám původní matici A rozloží na součin dolní trojuhelníkové matice a horní trojuhelníkové matice. LU rozklad je založen na Gaussově eliminaci následovně
$$A \longrightarrow A^{(1)} \equiv \begin{bmatrix}
    a_{1,1} & a_{1,2} & \hdots & a_{1,n} \\
    0 & a_{2,2}^1& \hdots & a_{2,n}^1 \\
    \vdots & \vdots & \ddots & \vdots \\
    0 & a_{m,2}^1 & \hdots & a_{m,n}^1 
\end{bmatrix} = M^{-1}A,$$
kde 
$$M_1^{-1}=I - m_1 e_1^T = \begin{bmatrix}
    1 & & & & \\[5pt]
    -\frac{a_{2,1}}{a_{1,1}} & 1 & & & \\[5pt]
    -\frac{a_{3,1}}{a_{1,1}} & & 1 & & \\[5pt]
    \vdots & & & \ddots & \\[5pt]
    -\frac{a_{m,1}}{a_{1,1}} & & & & 1\\
\end{bmatrix},\;\;
m_1=\begin{bmatrix}
    \frac{a_{2,1}}{a_{1,1}}\\
    \frac{a_{3,1}}{a_{1,1}}\\
    \vdots
    \frac{a_{m,1}}{a_{1,1}}\\
\end{bmatrix}.$$

Jelikož A má hodnost m, tak permutováním zaručíme, že nikdy nedělíme 0, tedy LU rozklad lze provést.
V  i-tém kroku bude
$$A^{(i)} \equiv \begin{bmatrix}
    a_{1,1} & a_{1,2} & \hdots & & & a_{1,n}\\
    0 & a_{2,2}^{(1)} & \hdots & & & a_{2,n}^{(1)}\\
    \vdots & 0 & \ddots & & &\\
      & \vdots & & a_{i,i}^{(i-1)} & \hdots & a_{i,n}^{(i-1)}\\
    0 & 0 & & a_{m,i}^{(i-1)} & \hdots & a_{m,n}^{(i-1)}

\end{bmatrix},\;\;
m_i=\begin{bmatrix}
    0\\
    \vdots\\
    0\\
    \frac{a_{i+1,i}}{a_{i,i}}\\
    \frac{a_{i+2,i}}{a_{i,i}}\\
    \vdots\\
    \frac{a_{m,i}}{a_{i,i}}\\
\end{bmatrix}.$$
Po m-1 krocích dostaneme 
$$
U \equiv A^{(m-1)} = M_{m-1}^{-1} \hdots M_{2}^{-1}M_{1}^{-1}A,\;\;
L \equiv M_1M_2\hdots M_{m-1},
$$
kde U je horní trojuhelníková typu $m \times n$ a L je dolní trojuhelníková řádu m s jednotkami na diagonále a upravením rovnosti dostáváme
$$A=LU=L[U_1 \; U_2],\;\; U_1=[u_1,\hdots,u_m],\;\; U_2=[u_{m+1},\hdots,u_n].$$

\subsubsection{Výpočet jádra matice pomocí LU rozkladu}
Nyní můžeme LU rozklad použít k řešení naší úlohy $AB=0$.
\begin{algoritmus}[Výpočet jádra pomocí LU rozkladu]
    Mějme matici A typu $m \times n$ hodnosti m.
    \begin{itemize}
        \item Provedeme potřebné permutace matice A na matici $P_1AP_2$, abychom měli rozklad matice v hledaném tvaru.
        \item Pomocí nějaké varianty vypočítáme LU rozklad, tedy $P_1AP_2=LU=L[U_1 \; U_2].$
        \item Položíme $J=U_1^{-1}U_2$, pak hledané řešení úlohy je
        \[B=P_2 \begin{bmatrix}
            -J\\
            I_{n-m}
        \end{bmatrix}
        \]
    \end{itemize}
\end{algoritmus}

\subsection{Sloupcová eliminace}
Metoda sloupcové eliminace je založena na stejné myšlence jako LU rozklad, nicméně místo poddiagonálních prvků budeme nulovat naddiagonální prvky. Tedy
$$
M_1^{-1} = \begin{bmatrix}
    1 & -\frac{a_{1,2}}{a_{1,1}} & -\frac{a_{1,3}}{a_{1,1}} & \hdots & -\frac{a_{1,n}}{a_{1,1}} \\
    0 & 1 & 0 & \hdots & 0 \\
    \vdots & & \ddots & & \\
    0 & & & & 1

\end{bmatrix},
$$
pak
$$
A^{(1)} = AM_1^{-1} \begin{bmatrix}
    a_{1,1} & 0 & 0 & \hdots & 0 \\
    a_{2,1} & a_{2,2}^{(1)} & a_{2,3}^{(1)} & \hdots & a_{2,n}^{(1)} \\
    \vdots & \vdots & \vdots & & \vdots \\
    a_{m,1} & a_{m,2}^{(1)} & \hdots & & a_{m,n}^{(1)}
\end{bmatrix}.
$$
V obecném i-tém kroku ($i<m$) budeme mít 
$$
M_i^{-1} = \begin{bmatrix}
    1 & 0 & \hdots & & & & 0 \\
    0 & 1 & & & & & 0 \\
    \vdots & 0 & \ddots & & & \\
    & & & 1 & -\frac{a_{i,i+1}}{a_{i,i}}& \hdots & -\frac{a_{m,n}}{a_{i,i}} \\ 
    &  & & & \ddots & & 0 \\
    &  & & & & & \vdots \\
    0 & \hdots & & & & 0 & 1

\end{bmatrix}
$$
a 
$$
A^{(i)}=AM_1^{-1} M_2^{-1} \hdots M_i^{-1} = 
$$
$$=
\begin{bmatrix}
    a_{1,1} & 0 & \hdots & & & & & &0 \\
    a_{2,1} & a_{2,2}^{(1)} & 0 & \hdots & & & & & \vdots  \\
    \vdots & a_{3,2}^{(1)} & a_{3,3}^{(2)} & \ddots & & &  \\
    & & \ddots & \ddots & & & & & \\
    & & & a_{i,i-1}^{(i-1)} & a_{i,i}^{(i)} & 0 & \hdots & \\
    & & & & \ddots & \ddots & & \\
    a_{m,1} & a_{m,2}^{(1)} & \hdots & & & & a_{m,m}^{(i)}& \hdots & a_{m,n}^{(i)}

\end{bmatrix}.
$$

Po m-1. kroku tedy dostáváme rozklad
\[P_1AP_2=[L \; 0]U, U^{-1} = [N_1 \; N_2],\]
kde $N_1$ je typu $n \times m$ a $N_2$ je typu $n \times (n-m)$, L je dolní trojuhelníková řádu m a U je horní trojuhelníková řádu n.
\subsubsection{Výpočet jádra matice pomocí sloupcové eliminace}
\begin{algoritmus}[Výpočet jádra pomocí sloupcové eliminace]
    Mějme matici A typu $m \times n$ hodnosti m.
    \begin{itemize}
        \item Permutováním převedeme A na $P_1AP_2.$
        \item Provedeme analogii Gaussovy eliminace pro sloupcovou eliminaci a získáme rozklad $P_1AP_2=[L \; 0] U.$
        \item Položíme $U^{-1}=[N_1 \; N_2]$, potom hledaná matice nulového prostoru je
        \[B=P_2N_2.\]
    \end{itemize}
    
\end{algoritmus}

\subsection{Gauss-Jordanova eliminace}
Tato metoda navazuje na LU rozklad, máme-li tedy rozklad \[P_1AP_2=L[U_1 \; U_2],\] tak v Gauss-Jordanově eliminaci ještě vynulujeme naddiagonální prvky matice $[U_1 \; U_2]$ s tím, že před každým krokem vynormujeme příslušný diagonální prvek tak, aby byl roven 1.

Postup v i-tém kroce je následující
\[
M_i^{-1} = \begin{bmatrix}
    1 & 0 & \hdots & 0 & -a_{1,i}&0 &\hdots & 0 \\
    0 & 1 & \ddots & & \vdots & & & \vdots   \\
    \vdots & 0 & \ddots & & -a_{i-1,i} & & \\
    & & \vdots & & 1 & 0 & \\
    & & & & & \ddots & \ddots& \\
    & & & & & & 1 & 0 \\
    0& \hdots & & & & & 0 & 1
\end{bmatrix}, \; A^{(i)} = \frac{1}{a_{i,i}^{(i-1)}}M_i^{(-1)}A^{(i-1)}
\]
Po m-1. kroku dostáváme
\[ A^{(m-1)} = M_{m-1}^{-1} \hdots M_2^{-1}M_1^{-1}L[U_1\; U_2] = \begin{bmatrix}
    1 & 0 & \hdots & 0 & a_{1,m+1}^{(m-1)} & \hdots & a_{1,n}^{(m-1)} \\
    0 & 1 & & \vdots & a_{2,m+1}^{(m-1)} & \hdots & a_{2,n}^{(m-1)} \\
    \vdots & 0 & \ddots & & \vdots & & \vdots \\
    0 & \hdots & 0 & 1 & a_{m+1,m+1}^{(m-1)} & \vdots & a_{m,n}^{(m-1)}
\end{bmatrix}
= [I_m \; J]
\]
kde
\[J=\begin{bmatrix}
    a_{1,m+1}^{(m-1)} & a_{1,m+2}^{(m-1)} & \hdots & a_{1,n}^{(m-1)} \\
    a_{2,m+1}^{(m-1)} & a_{2,m+2}^{(m-1)} & \hdots & a_{2,n}^{(m-1)} \\
    \vdots & &\ddots & & \\
    a_{m+1,m+1}^{(m-1)}& \hdots & & a_{m,n}^{(m-1)}
\end{bmatrix}
\]
Získali jsme tedy následující rozklad 
\[ P_1 A P_2 = M [I_m \; J]
,\]
kde M je řádu m a J je typu $m \times (n-m)$.
\subsubsection{Výpočet jádra pomocí Gauss-Jordanovy eliminace}
\begin{algoritmus}[Výpočet jádra pomocí Gauss Jordanovy eliminace]
    Mějme matici A typu $m \times n$ hodnosti m.
    \begin{itemize}
        \item Provedeme potřebné permutace a převedeme matici A na matici $P_1AP_2.$
        \item Vypočítáme LU rozklad matice $P_1AP_2=LU.$
        \item Postupným přenásobováním a sloupcovou eliminací získáme požadovaný rozklad $ P_1 A P_2 = M [I_m \; J]$
        \item Hledaný nulový prostor matice A je roven \[B=P_2 \begin{bmatrix}
            -J\\
            I_{n-m}
        \end{bmatrix}.\]
    \end{itemize}
    
    
\end{algoritmus}

\subsection{QR rozklad}
Tento rozklad nám rozloží matici A na součin unitární matice Q typu $m \times m$ a horní trojuhelníkové matice R typu $m \times n$, tedy $A=QR.$
Máme více algoritmů pro výpočet QR rozkladu matice A, konkrétně vybíráme mezi metodami založených na Givensových rotacích, Householderových reflexích a nebo Gram-Schmidtovým ortogonalizačním procesem.V našem případě budeme uvažovat Gram-Schmidtův ortogonalizační proces.Ten nám z matice A iteračně vygeneruje ortonormální bázi $Q=(q_1,\hdots,q_m)$ a matici koeficientů R.

Označme $A=[a_1,a_2,\hdots,a_n]$, kde $a_1,\hdots,a_m$ jsou sloupcové vektory matice A, budeme navíc v každém kroku iterace chtít, aby platilo $span\{a_1,\hdots,a_k\}=span\{q_1,\hdots,q_k\}.$ Postup je pak následující
\begin{itemize}
    \item V prvním kroku normováním dostaneme $r_{1,1}= \left \lVert a_1 \right \rVert$ a $q_1 = \frac{a_1}{r_{1,1}}$.
    \item V druhém kroku musí platit $span\{a_1,a_2\}=span\{q_1,q_2\}$ a $q_1 \perp q_2$, což splníme tím, že odečteme projekci vektoru $a_2$ do podprostoru $span{q_1}$ a vynormujeme. Tedy
    \[z=(I - q_1q_1^*)a_2=a_2-(q_1^*a_2)q_1, \;\; r_{1,2}= q_1^*a_2, \;\; r_{2,2}= \left \lVert z \right \rVert,\]
    \[q_2=\frac{z}{r_{2,2}}.\]
    \item V k-tém kroku stejným postupem odečteme od od vektoru $a_k$ jeho projekci do prostoru $span{q_1,\hdots,q_{k-1}}$ a vynormujeme. Konkrétně
    \[z=(I-Q_{k-1}Q_{k-1}^*)a_k=a_k - \sum_{i=1}^{k-1}(q_i^*a_k)q_i,\;\; r_{i,k}=q_i^*a_k,\; k=1,\hdots,k-1,\;\; r_{k,k}=\left \lVert z \right \rVert,\]
    \[q_k=\frac{z}{r_{k,k}}.\]
\end{itemize}
Ve výsledku dostáváme unitární matici $Q=[q_1,\hdots,q_m]$ a horní trojuhelníkovou matici typu $m \times n$
\[R=\begin{bmatrix}
    r_{1,1} & r_{1,2} & \hdots & r_{1,m} & r_{1,m+1} & \hdots & r_{1,n} \\
    0 & r_{2,2} & & r_{2,n} & r_{2,m+1} & & r_{2,n} \\
    \vdots & \ddots & \ddots & & &\vdots \\
    0 & \vdots & 0 & r_{m,m} & r_{m,m+1} & & r_{m,n}
\end{bmatrix}\]
a bude platit rovnost
\[A=QR.\]
Ve výpočtu výše jsme předpokládali, že prvních m sloupcových vektorů matice A jsou lineárně nezávislé, což můžeme, neboť matice A je hodnosti m. Permutováním tedy můžeme tyto předpoklady zaručit.

\subsubsection{Výpočet jádra pomocí QR rozkladu}
Mějme matici A typu $m \times n$ hodnosti m.
\begin{algoritmus}[Výpočet jádra pomocí QR rozkladu]
    Mějme matici A typu $m \times n$ hodnosti m.
    \begin{itemize}
        \item Permutováním převedeme A na $P_1AP_2.$
        \item Pomocí vhodné metody vypočítáme QR rozklad $P_1AP_2=QR=Q[R_1 \; R_2].$
        \item Položíme $J=R_1^{-1}R_2.$ Potom hledaná matice nulového prostoru je \[B=P_2 \begin{bmatrix}
            -J\\
            I_{n-m}
        \end{bmatrix}.\]
    \end{itemize}
    kde pomocí $R=[r_1,\hdots,r_m,r_{m+1},\hdots,r_n]$ značíme $R_1=[r_1,\hdots,r_m],\;\; R_2=[r_{m+1},\hdots,r_n].$
\end{algoritmus}
\subsection{LQ rozklad}
Tento rozklad rozloží matici A na součin
\[A = [L\; 0]Q=[L\; 0] \begin{bmatrix}
    Q_1\\
    Q_2
\end{bmatrix},\]
kde L je dolní trojuhelníková typu $m \times m$ a Q je unitární typu $n \times n$, $Q_1$ je typu $m \times n$ a $Q_2$ je typu $(n-m)\times n.$
LQ rozklad pomocí unitárních transformací aplikovaných zprava nuluje naddiagonální prvky matice A, tedy
\[AQ=[L\; 0] \iff A=[L\; 0] Q^*.\]
\subsubsection{Vypočet jádra pomocí LQ rozkladu}
\begin{algoritmus}{Vypočet jádra pomocí LQ rozkladu}
    \begin{itemize}
        Mějme matici A typu $m \times n.$
        \item Provedeme potřebné permutace a převedeme matici A na matici $P_1AP_2.$
        \item Vypočítáme pomocí vhodně zvolené metody rozklad $P_1AP_2=[L\; 0]Q = [L\; 0] \begin{bmatrix}
            Q_1\\
            Q_2
            \end{bmatrix}$
        \item Hledaná matice nulového prostoru je 
        \[B=P_2Q_2.\]
    \end{itemize}
    
\end{algoritmus}


\section{Vlastnosti metod}
V této části se podíváme na vlastnosti uvedených metod, konkrétně na počet provedených operací, pamětové náročnosti a jejich numerické vlastnosti. Konkrétně se pokusíme identifikovat, v jakých situacích by u daných metod mohly vznikat komplikace.

\subsection{Gauss-Jordanova eliminace}
Jelikož cílem této metody je dostat jednotkovou matici na diagonále, tak tato metoda bude potřebovat velký počet eliminací.
Pokud budeme uvažovat A řídkou matici, tak v procesu eliminace bude docházet k zaplnění a ačkoliv eliminačním procesem prvky vynulují, tak je třeba uchovat tyto nově nenulové prvky v paměti.
Pro rozklad $P_1AP_2=M[I_m\; J]$ je matice J obvykle hustá a bude zabírat velké množství paměti pro velké n-m.
Může také dojít k numerickým problémům s pivotací, například už i pro Gaussovu eliminaci pro matici řádu n platí
\[\begin{bmatrix}
    1 & & & & 1 \\
    -1 & 1 & & & 1 \\
    \vdots & \ddots & \ddots &  & \vdots \\
    -1 & \hdots & -1 & 1 & 1 \\
    -1 & \hdots & -1 & -1 & 1 \\   
\end{bmatrix}
\longrightarrow
\begin{bmatrix}
    1 & & & & 1 \\
    0& 1 & & & 2 \\
    \vdots & \ddots & \ddots & & \vdots \\
    0 & \hdots & 0 & 1 & 2^{n-2} \\
    0 & \hdots & 0 & 0 & 2^{n-1}
\end{bmatrix}\]
\end{document}
